\chapter{DSO代码分析--前端跟踪}

整个项目代码量比较大,文件很多。而且作者为了提高计算效率,优化相关的部分,很多中间结果共用,也没有采用开源的优化库ceres、gtsam等,所有部分纯手撸。而且还进行了SSE加速,也没使用OpenCV中的算法。因此,代码及其复杂,还涉及到分厂多的参数,这些参数与作者的使用调试经验强相关。
整个关键代码部分,可以分为三个部分:
\begin{itemize}
	\item  初始化
	\item  帧间跟踪
	\item  局部优化
\end{itemize} 
\paragraph{初始化} 作者在github的项目主页上说到,初始化并不是很鲁棒,用户可以自己实现自己的初始化操作。
\paragraph{帧间跟踪} 帧间跟踪主要是进行图像对齐,计算帧间的pose,类似与特征点法这种的计算E或者F,然后分解得到Pose的过程。
\paragraph{局部优化} 局部优化中做的事儿就有点儿多了,重头戏。


%===================================================
\section{标定矫正}


%===================================================
\section{系统初始化}


%===================================================
\section{帧间跟踪}	
\vspace{0.2em}
DSO中帧间跟踪时,只计算当前帧与前一个关键帧间的pose。
\indent 完成粗跟踪的类是CoarseTracker,该类在FullSystem中共有两个对象,分别是coarseTracker\_forNewKF、coarseTracker。\\
二者的区别是:\\
\textbf{coarseTracker\_forNewKF}:更新存储关键帧的信息,包括refFrameID信息,在makeKeyFrame()函数中完成,且是在完成关键帧之后和边缘化之前;\\
\textbf{coarseTracker}:所有帧的跟踪,新帧到来时,若coarseTracker\_forNewKF中的refFrameID被更新(即是否比coarseTracker中的大),就将coarseTracker\_forNewKF赋给coarseTracker。若没有,coarseTracker维持之前的状态,即最后一个keyframe。





%%%%%%%%%%%%%%%%%%%%%%%%%%%%%%%%%%%%%%%%%%%%%%%%%%%%%%%%%%%%%%%%%%%%%%%%%%%%%%%%%%%%%%%%%%%%%%%%%%%%%%%%%%%%%%%
%%%%%%%%%%%%%%%%%%%%%%%%%%%%%%%%%%%%%%%%%%%%%%%%%%%%%%%%%%%%%%%%%%%%%%%%%%%%%%%%%%%%%%%%%%%%%%%%%%%%%%%%%%%%%%%

\chapter{DSO代码分析--后端优化}
后端优化在makeKeyFrame()中,依次有十几项步骤:



%=================================================================================
\section{点的深度计算}
关键点的深度的计算都在后端完成,前端之负责计算两帧间的pose的初值。分为两步:
\begin{itemize}
	\item [1)] 通过深度滤波器更新点的深度范围,直到点的深度收敛,作为深度的初值。
	\item [2)] 在完成帧间pose的优化后,进一步优化点的深度值。
\end{itemize} 


\subsection{深度滤波}
通过深度滤波器完成深度范围的更新,在函数ImmaturePoint::traceOn()中,主要维持两个变量:idepth\_min、idepth\_min。将点投影到Target帧中,形成对应的两个点ptpMin、ptpMax,如果两个点的距离小于一定的阈值,则认为该点深度收敛:
\begin{lstlisting}[caption={  ImmaturePoint::traceOn()}]  
Vec3f pr = hostToFrame_KRKi * Vec3f(u,v, 1);
Vec3f ptpMin = pr + hostToFrame_Kt*idepth_min;		
.....
Vec3f ptpMax;
ptpMax = pr + hostToFrame_Kt*idepth_max;
......
if(!(uMax > 4 && vMax > 4 && uMax < wG[0]-5 && vMax < hG[0]-5))
{
  if(debugPrint) printf("OOB uMax  %f %f - %f %f!\n",u,v, uMax, vMax);
  lastTraceUV = Vec2f(-1,-1);
  lastTracePixelInterval=0;
  
  return lastTraceStatus = ImmaturePointStatus::IPS_OOB;
}
\end{lstlisting}  

\noindent 对深度进行深度更新:
\begin{equation}
\begin{aligned}
u-\delta e dx &= \frac{(pr + \rho_{min}Kt)[0]}{(pr + \rho_{min}Kt)[2]}\\
u+\delta e dx &= \frac{(pr + \rho_{max}Kt)[0]}{(pr + \rho_{max}Kt)[2]}
\end{aligned}
\end{equation}
式中,$pr=KR_{TH}K^{-1}*p_{H}$是点从Host帧中旋转到Target帧后的样子(即Point Rotation的简写),未考虑平移。平移和深度相关,深度相关的和前面重点强调的是一样的。
变形后得到代码中的形式:
\begin{equation}
\begin{aligned}
\rho_{min}  &= \frac{(pr[2]*(u-\delta e dx)-pr[0]}{(kt)[0] - (Kt)[2]*(u-\delta e dx)}\\
\rho_{max}  &= \frac{(pr[2]*(u+\delta e dx)-pr[0]}{(kt)[0] - (Kt)[2]*(u+\delta e dx)}
\end{aligned}
\end{equation}

\subsection{深度优化}
在优化时,会专门对深度进行优化


%==============================================================================================================
\section{Bundle Adjustment}
\indent 这一部分是整篇中最为核心的地方。作者将Hessian矩阵分为active、linear、marginal三部分。
在AccumulatedTopHessianSSE::addPoint()函数中,模板参数mode的值0、1、2分别表示active、linearized、marginalize。但是经测试linearized下面的实际部分并未执行,作者在逻辑中将其屏蔽掉了。
\subsection*{active部分}

\subsection*{marginal部分}

\section{边缘化}


\chapter{代码细节}
\section{比例因子}
代码中好多地方出现比例因子,这些因子作用是提高求解方程式 $$ H\bigtriangleup\!x=b $$的数值稳定性和精确度.\\
出现的地方:\\
1、方程求解的过程中\\
2、FrameHessian中\\
